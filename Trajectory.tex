\chapter{Trajectory Planning}
\label{cha:trajectory}
For the two most advanced modes, i. e. the Half-Automatic and the Full-Automatic Mode, trajectories had to be generated. In this chapter the best trajectories for \textsc{Skye} are elaborated.

%\subsection{Our Approach}
%From the GUI it was given that the goal trajectory would be a multipoint-interpolating %trajectory. The user is able to define waypoints on a map which afterwards should be %connected with a reasonable and realizable trajectory. Beside interpolating trajectories %there exist also approximating trajectories but they were not taken into consideration, since %usually the user wants skye fly directly through a waypoint.

%In another Bsc Thesis elaborated in this project a controller for waypoint following was %designed. So it was convenient in the scope of this Thesis to use this controller instead of %a specialized trajectory controller.

\section{Experimental Design}

\section{Definition of Trajectories}
\label{sec:definition}
\subsection{Paths and Trajectories}



\subsection{Interpolation and Approximation}
If one wants to draw a line through a set of data points, there exists two ways to do this. On the one hand the line must pass all data points no matter how many bends it will have, on the other hand the line tries to best fit the data, i.e. a function of a  certain order is adopted to best fit the date. E.g. this can be done with least-squares. If the pilot  defines the trajectory with a set of waypoints, i.e. data points, he usually wants the (UAV already defined?!!!!) to pass through all of them. Therefore the waypoints must be interpolated and not approximated with a suitable curve.

\textbf{(!!!Bsp Plot von interpolierenden und approximierenden funktionen!!!)}


\section{Spline Theory}
A set of data points can be interpolated with one single curve or with a set of curves defined over a certain interval.  For a 
\label{sec:splineTheory}
references to \cite{engeln}, \cite{biagiotti} and \cite{doessegger}
\subsubsection{Continuity}
\subsubsection{Boundary Conditions}
\subsubsection{Polynomial Order}
\subsubsection{Parametrization}
\subsection{Piecewise Polynomial Interpolating Splines}
\subsubsection{Boundary Conditions}
\subsubsection{Polynomial Order}
\subsubsection{Parameterization}
\subsection{B-Splines}
\subsubsection{Boundary Conditions}
\subsubsection{Polynomial Order}
\subsubsection{Parametrization}

\section{Trajectory Generation}
\label{sec:trajectoryGeneration}
\subsection{System Constraints}
\subsubsection{Maximum Velocities and Accelerations}
In order to plan a feasible trajectory one has to know the capabilities of the system. Here just a basic derivation for the velocities and accelerations is given, for more details refer to (!!!!Bsc Thesis Joe, Bsc Thesis Andy)\\

The maximum feasible acceleration in any direction is calculated to be:

\begin{equation}
  \left|a_{max} \right| =  \frac{\left|F_{res, w}\right|}{m_{tot}} = 0.96 m/s^2
\end{equation}

Whereas the $F_{res,w}$ is the force resulting from all four thrusters operated under full load in the worst direction and $m_{tot}$ is the sum of the masses of the helium, the virtual mass and the mass of the system itself.\\


The maximum feasible velocity in any direction is calculated to be:

\begin{equation}
\left|v_{max} \right| = \sqrt{\frac{\left|F_{res,w} \right|}{\frac{1}{2}c_d \rho \pi r^2}}=2.9 m/s
\end{equation}

which is nothing but $ \left|F_{res,min} \right| = \left|F_{dray} \right| $.\\

For trajectories for position and orientation the maximal feasible angular acceleration is also important. It is calculated to be:

\begin{equation}
  \left|\Psi_{max} \right| =  \frac{\left|M_{res,w}\right|}{\left| \lambda_{max, J_{B}} \right|} = 2.06 rad/s^2 
\end{equation}

which is quite conservative because it is assumed that worst axis for turning is also the principle axis of the inertia tensor with the highest inertia.\\

Since the system is almost undamped for rotations, the rotational velocities will never be the limiting factor.

\subsection{Time Parametrization}

\section{Controller Implementation}
\label{sec:controllerImplementation}
\subsection{Trajectory Controller}
see \cite{snider} and \cite{deluca}
\subsection{Pure Pursuit Position Controller}
see also \cite{snider}
\subsection{Cross Track Error Controller}
see \cite{williams}

\section{Discussion}
\label{sec:discussion}

