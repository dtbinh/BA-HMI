\chapter{Trajectory Planning}
For the two most advanced modes, i. e. the Half-Automatic and the Full-Automatic Mode, trajectories had to be generated. In this chapter the best trajectories for skye are elaborated.

\section{Introduction}

\subsection{Definition}
What is a trajectory...(notation, parameter, time...)
How do we intend to realize our idea... 

\subsection{Our Approach}
From the GUI it was given that the goal trajectory would be a multipoint-interpolating trajectory. The user is able to define waypoints on a map which afterwards should be connected with a reasonable and realizable trajectory. Beside interpolating trajectories there exist also approximating trajectories but they were not taken into consideration, since usually the user wants skye fly directly through a waypoint.

In another Bsc Thesis elaborated in this project a controller for waypoint following was designed. So it was convenient in the scope of this Thesis to use this controller instead of a specialized trajectory controller.

\section{Vorschlag: Geometric Parameterized Paths}
BLA: Everything considering generating splines, boundery conditions, order, and comparision and evaluation (skye independent)
\section{Vorschlag: Time Parameterized Trajectories}
BLA: System constraints, time parametrization
\section{Vorschlag: Trajectory Tracking}
BLA: Feeding the trajectories into skye, controller appoach, evaluation (comparision trajectory and trace) \\
\textsc{Unsch"on:} Hier muss alles was oben theoretisch beschrieben wird repetiert werden (BC, Order, Constraints?, different time parametrization) \\
The goal of tracking is

\section{System Constraints for Trajectory}

\subsection{Maximum Velocities and Accelerations}
In order to plan a feasible trajectory one has to know the capabilities of the system. Here just a basic derivation for the velocities and accelerations is given, for more details refer to (!!!!Bsc Thesis Joe, Bsc Thesis Andy)\\

The maximum feasible acceleration in any direction is calculated to be:

\begin{equation}
  \left|a_{max} \right| =  \frac{\left|F_{res, w}\right|}{m_{tot}} = 0.96 m/s^2
\end{equation}

Whereas the $F_{res,w}$ is the force resulting from all four thrusters operated under full load in the worst direction and $m_{tot}$ is the sum of the masses of the helium, the virtual mass and the mass of the system itself.\\


The maximum feasible velocity in any direction is calculated to be:

\begin{equation}
\left|v_{max} \right| = \sqrt{\frac{\left|F_{res,w} \right|}{\frac{1}{2}c_d \rho \pi r^2}}=4.7 m/s
\end{equation}

which is nothing but $ \left|F_{res,min} \right| = \left|F_{dray} \right| $.\\

For trajectories for position and orientation the maximal feasible angular acceleration is also important. It is calculated to be:

\begin{equation}
  \left|\Psi_{max} \right| =  \frac{\left|M_{res,w}\right|}{\left| \lambda_{max, J_{B}} \right|} = 2.82 rad/s^2 
\end{equation}

which is quite conservative because it is assumed that worst axis for turning is also the principle axis of the inertia tensor with the highest inertia.\\

Since the system is almost undamped for rotations, the rotational velocities will never be the limiting factor.





\subsection{Continuity}







