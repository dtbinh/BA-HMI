%!TEX root = Bericht.tex
\chapter{Conclusion}
\textit{
What is the strongest and most important statement that you can make from your observations? 
If you met the reader at a meeting six months from now, what do you want them to remember about your paper? 
Refer back to problem posed, and describe the conclusions that you reached from carrying out this investigation, summarize new observations, new interpretations, and new insights that have resulted from the present work.
Include the broader implications of your results. 
Do not repeat word for word the abstract, introduction or discussion.}\\
%\textit{To summarize
%� What you researched� Nature of your main arguments� How you researched it� What you discovered� What pre-existing views were challenged
%2. To provide an overview of 
%The new knowledge or information discovered� The significance of your research (where is it new?)� The limitations of your thesis (concepts, data)� Speculation on the implications of these limitations� Areas for further development and research(alternative data sets; links with other fields; differentmethod applied to same data)
%The �1 Must� and the �3 Ideals�
%You
%must 1. Make a clear and concise statement of theoriginal contribution to knowledge found in your thesis.Ideally 
%you should aspire to1. Show links between the key ideas spread acrosschapters2. Show your commitment to and enthusiasm for academic research3. Leave a positive impression with the examiner .\\
%Avoid claiming findings that you have not proventhroughout your thesis� Avoid introducing new data� Avoid hiding weaknesses or limitations in your thesis(make a virtue of showing strong analytical skills and self-critique by discussing the limitations--but don�t gooverboard on this!)�  Avoid making practical recommendations (e.g. for policy).If you must include them put them in an appendix.� Avoid being too long (repetitive) or too short (sayingnothing of importance)
%Sample Conclusion Structure (1)
%One paragraph stating what you researched and what your original contribution to the field is�then break into sections� One section on what you researched and how you did it� One section on what are the main findings were� showinglinks across chapters (this explains why you chose thestructure you did)� One section on possible areas for future research� Final section reminding readers of the original contributionand significance of your research to your field.
%Sample Conclusion Structure (2)
%One paragraph stating what you researched and what your original contribution to the field is�then break into sections� Main finding 1 (subsections on how you arrived at thisfinding and how it challenges previous research)� Main finding 2 (subsections on how you arrived at thisfinding and how it challenges previous research)� Main finding 3 (subsections on how you arrived at thisfinding and how it challenges previous research)� One section on potential leads to openings for further research resulting from your research� Final section reminding readers of the original contributionand significance of your research to your field.
%Concluding Conclusion
%Be enthused by the prospect of writing your conclusion� �It�s downhill from here�!�To avoid the concluding chapter shock�
%� Start writing notes of content for potentialsections of your conclusion as you write other chapters�add bits in as you think of them� If you find a succinct quote that encapsulatescommon misunderstandings about your fieldthen use it at the start to bounce off.
%}
%
