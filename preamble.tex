	%!TEX root = Bericht.tex
%---------------------------------------------------------------------------
% Preface

%\chapter*{Vorwort}

%Bla bla \dots

 %\cleardoublepage

%---------------------------------------------------------------------------
% Table of contents

 \setcounter{tocdepth}{2}
 \tableofcontents

 \cleardoublepage

%---------------------------------------------------------------------------
% Abstract

%\chapter*{Zusammenfassung}
% \addcontentsline{toc}{chapter}{Zusammenfassung}

%Bla bla \dots

% \cleardoublepage

\chapter*{Abstract}
 \addcontentsline{toc}{chapter}{Abstract}

This Bachelors thesis was written within project \textsc{Skye}, whose goal is to create a high agile spherical blimp called \textsc{Skye}. Main tasks of this novel system are image capturing and entertainment. The four tetrahedrally arranged motors of the system can be rotated to thrust into any tangential direction to the hull. Combined with the symmetrical design this allows to control all motions in the three dimensional space. A camera system consisting of a high quality camera as well as two additional low size cameras can be used to capture and store images onboard or transmit them to a ground station. \\
To control the six degrees of freedom (DOF), manual control modes as well as automatic control and a combined manual and automatic control mode have been developed within this thesis. A combination of a tablet PC with a 3D mouse proofed to be suitable to provide intuitive control. A GUI has been developed that easily allows to switch between control modes, to display the system's status and even to control the system via a touch interface. The touch input screen is underlaid by a map for translations and the video stream for rotations respectively. For automatic control, trajectories for \textsc{Skye} have been designed and compared. Three approaches to track them have been implemented and tested in a simulation environment.

 \cleardoublepage

%---------------------------------------------------------------------------
% Acknowledgements

%\chapter*{Acknowledgements}\label{chap:Acknowledgements}
% \addcontentsline{toc}{chapter}{Acknowledgements}
\chapter*{Acknowledgements}
 \addcontentsline{toc}{chapter}{Acknowledgements}

Without the help of many individuals this thesis would not have been possible. We received the necessary support from all sides throughout the project to realize this HMI. We would like to thank everybody who helped us during its development and therefore also contributed to the successful rollout where \textsc{Skye} flew smoothly piloted with this HMI through the \textsc{ETH} main building.
\\
\\
Special thanks to Prof. Dr. Roland Y. Siegwart, who gave us the opportunity to do this thesis within project \textsc{Skye} at his institute, the \textsc{ASL}.
\\
\\
Also special thanks to Dr. Paul Beardsley, who constantly motivated us and inspired us with great ideas for an intuitive HMI.
\\
\\
We would also like to express our gratitude and thanks to our supervisors, the PhD students Konrad Rudin and Javier Alonso Mora, for the valuable guidance and advice throughout this thesis.
\\
\\
Out of many more individuals, we would like to mention by name Lorenz Meier, Gerhard R{\"o}thlin and Alexander Rudyk. They especially helped us with their programming skills when having reached an impasse.
\\
\\
Of course this thesis was only possible because of the whole team \textsc{Skye}. Thank you very much! It was a pleasure to work with all of you.
\\
\\
Finally we would like to express our deepest gratitude to our families and beloved ones, Marina and Eliane. They gave us encouragement and showed great patience when it was most required.
\\
\\
\\
\\
\\
We gained a lot of learning experience in this thesis and it was of real pleasure to see \textsc{Skye} finally fly with the HMI developed in this thesis.
\\
\\
\\
\\
Z{\"u}rich, June 2012\\
\\
Matthias Krebs\\
Anton Ledergerber\\





 \cleardoublepage

%---------------------------------------------------------------------------
% Symbols

%\chapter*{Symbolverzeichnis}\label{chap:symbole}
% \addcontentsline{toc}{chapter}{Symbolverzeichnis}
\chapter*{Symbols}\label{chap:symbole}
 \addcontentsline{toc}{chapter}{Symbols}

%\section*{Symbole}
\section*{Symbols}
\begin{tabbing}
 \hspace*{3cm} \= \kill
  ${\bf bold}$			\> Vector \\[0.5ex]
  $\left| \cdots \right|$		\> Absulute value \\[0.5ex]
  $\| \cdots \|$		\> Euler norm \\[0.5ex]
  ${\bf q}$				\> 3 dimensional waypoint vector \\[0.5ex]
  ${\bf r}(t)$			\> 3 dimensional position state vector \\[0.5ex]
  ${\bf p}(u)$			\> 3 dimensional path vector \\[0.5ex]
  ${\bf \tilde{p}}(t)$	\> 3 dimensianal trajectory vector \\[0.5ex]
  $u$ 					\> Parameter for path \\[0.5ex]			
  $t$					\> Parameter for trajectory; time \\[0.5ex]	
  $L_p$					\> Geometrical length of path or trajectory	\\[0.5ex]
  $T_p$					\> Time	length of trajectory \\[0.5ex]
  $G^h$					\> Geometrical continuity of a curve \\[0.5ex]
  $C^h$					\> Parametric continuity of a curve \\[0.5ex]
  $n+1$						\> Number of waypoints \\[0.5ex]
  $n_{knots}+1$			\> Number of knots \\[0.5ex]
 \end{tabbing}

%\section*{Indizes}
\section*{Indices}
\begin{tabbing}
 \hspace*{1.6cm}  \= \kill
 $cl$ 				\> Closest point on path or trajectory \\[0.5ex]
 $k$ 				\> Index of waypoint $0 \leq k \leq n$ \\[0.5ex]
\end{tabbing}

%\section*{Akronyme und Abk�rzungen}
\section*{Acronyms and Abbreviations}
\begin{tabbing}
 \hspace*{1.6cm}  \= \kill
% ETH \> Eidgen�ssische Technische Hochschule \\[0.5ex]
% ASL \> Autonomous Systems Lab \\[0.5ex]
% HMI \> Human-Machine Interface \\[0.5ex]
% GUI \> Graphical User Interface \\[0.5ex]
% DOF \> Degree of Freedom \\[0.5ex]
% PX4FMU \> \textsc{Pixhawk} Flight Management Unit \\[0.5ex]
% IMU \> Inertial Measurements Unit \\[0.5ex]
% GPS \> Global Positioning System \\[0.5ex]
% RC  \>	Remote Control \\[0.5ex]
% NED \> North-East-Down \\[0.5ex]
% API \> Application Programming Interface \\[0.5ex]
% AC  \> Assisted Control Mode \\[0.5ex]
% DC  \> Direct Control Mode \\[0.5ex]
% HAC \> Half Automatic Control Mode \\[0.5ex]
% FAC \> Full Automatic Control Mode \\[0.5ex]

% Alphabetical ordered
 AC  \> Assisted Control Mode \\[0.5ex]
 API \> Application Programming Interface \\[0.5ex]
 ASL \> Autonomous Systems Lab \\[0.5ex]
 DC  \> Direct Control Mode \\[0.5ex]
 DOF \> Degree of Freedom \\[0.5ex]
 ETH \> Eidgen�ssische Technische Hochschule \\[0.5ex]
 FAC \> Full Automatic Control Mode \\[0.5ex] 
 GPS \> Global Positioning System \\[0.5ex]
 GUI \> Graphical User Interface \\[0.5ex]
 HAC \> Half Automatic Control Mode \\[0.5ex]
 HMI \> Human-Machine Interface \\[0.5ex]
 IMU \> Inertial Measurements Unit \\[0.5ex]
 NED \> North-East-Down \\[0.5ex]
 PX4FMU \> \textsc{Pixhawk} Flight Management Unit \\[0.5ex]
 RC  \>	Remote Control \\[0.5ex]
 
\end{tabbing}

 \cleardoublepage

%---------------------------------------------------------------------------
