%!TEX root = Bericht.tex
%---------------------------------------------------------------------------
% Preface

%\chapter*{Vorwort}

%Bla bla \dots

 %\cleardoublepage

%---------------------------------------------------------------------------
% Table of contents

 \setcounter{tocdepth}{2}
 \tableofcontents

 \cleardoublepage

%---------------------------------------------------------------------------
% Abstract

%\chapter*{Zusammenfassung}
% \addcontentsline{toc}{chapter}{Zusammenfassung}

%Bla bla \dots

% \cleardoublepage

\chapter*{Abstract}
 \addcontentsline{toc}{chapter}{Abstract}

This Bachelors thesis was written within project \textsc{Skye}, which is about a high agile spherical blimp called \textsc{Skye} . Main tasks of this novel system will be image capturing and entertainment. The four tetrahedrally arranged motors of the system can be rotated to thrust into any tangential direction to the hull. With the 8 actuation DOF, the 6 state DOF of \textsc{Skye} can be controlled independently. A camera system consisting of a high quality camera as well as two additional low size cameras for 3D reconstructions and live stream is placed on it. \\
To control the unusual freedom of 6DOF, manual control modes as well as automatic control and a combined manual and automatic control mode have been developed. A combination of a tablet PC with a 3DMouse proofed to be suitable to provide intuitive control. A GUI easily allows to switch between control modes, to display the system's status and even to control the system via a touch interface. The touch input screen is underlaid by a map for translations and the video stream for rotations respectively. For automatic control, optimal trajectories for \textsc{Skye} have been designed and compared. Three approaches to track them have been implemented and tested in a simulation environment.

 \cleardoublepage

%---------------------------------------------------------------------------
% Acknowledgements

%\chapter*{Acknowledgements}\label{chap:Acknowledgements}
% \addcontentsline{toc}{chapter}{Acknowledgements}
\chapter*{Acknowledgements}
 \addcontentsline{toc}{chapter}{Acknowledgements}

Without the help of a few people this thesis would not have been possible. We received the necessary support from all sides throughout the project to realize the this HMI.

Prof. Dr. Roland Y. Siegwart\\
Dr. Paul Beardsley\\
PhD students Konrad Rudin and Javier Alonso Mora\\
Further: \\
Gerhard R{\"o}thlin \\
Lorenz Meier \\
Alexander Rudyk\\



 \cleardoublepage

%---------------------------------------------------------------------------
% Symbols

%\chapter*{Symbolverzeichnis}\label{chap:symbole}
% \addcontentsline{toc}{chapter}{Symbolverzeichnis}
\chapter*{Symbols}\label{chap:symbole}
 \addcontentsline{toc}{chapter}{Symbols}

%\section*{Symbole}
\section*{Symbols}
\begin{tabbing}
 \hspace*{3cm} \= \kill
  ${\bf r}(t)$			\> State vector \\[0.5ex]
  ${\bf p}(u)$			\> 3 dimensional path \\[0.5ex]
  ${\bf \tilde{p}}(t)$	\> 3 dimensianal trajectory \\[0.5ex]
  $u$ 					\> Parameter for path \\[0.5ex]			
  $t$					\> Parameter for trajectory; time \\[0.5ex]	
  $L_p$					\> Geometrical length of path or trajectory	\\[0.5ex]
  $T_p$					\> Time	length of trajectory \\[0.5ex]
  $G^h$					\> Geometrical continuity of a curve \\[0.5ex]
  $C^h$					\> Parametric continuity of a curve \\[0.5ex]
  $n+1$						\> Number of waypoints \\[0.5ex]
  $n_{knots}+1$			\> Number of knots \\[0.5ex]
 \end{tabbing}

%\section*{Indizes}
\section*{Indices}
\begin{tabbing}
 \hspace*{1.6cm}  \= \kill
 $cl$ \> Closest point on path or trajectory \\[0.5ex]
 $lap$ \> Lookahead point on path or trajectory \\[0.5ex]
\end{tabbing}

%\section*{Akronyme und Abk�rzungen}
\section*{Acronyms and Abbreviations}
\begin{tabbing}
 \hspace*{1.6cm}  \= \kill
 ETH \> Eidgen�ssische Technische Hochschule \\[0.5ex]
 ASL \> Autonomous Systems Lab \\[0.5ex]
 HMI \> Human-Machine Interface \\[0.5ex]
 GUI \> Graphical User Interface \\[0.5ex]
 DOF \> Degree of Freedom \\[0.5ex]
\end{tabbing}

 \cleardoublepage

%---------------------------------------------------------------------------
